\documentclass[11pt]{article}
\usepackage[utf8]{inputenc}
\usepackage[dvipsnames]{xcolor}
\usepackage[margin=1in]{geometry}
\usepackage{amsfonts, amssymb, amsmath, amsthm, booktabs, hyperref, pgfplots, tikz}

\title{MATH 4100 Kritik Assignment 1 \\ \small \emph{Putting Students on the Path to Learning}}
\author{Joe Tran}
\date{\today}

\begin{document}

\maketitle

The article presented by Richard Clark, Paul Kirschner, and John Sweller introduces that after decades of research, they have found that students learn more effectively through direct and explicit instruction. It also mentions how teachers who provide minimal guidance expect the students, in a math class scenario, to understand the material independently, brainstorm in small groups to attempt to solve the problem, and then have the teacher provide the students with the solutions.

I was doing a lab activity in my chemistry class; the teacher used the minimal guidance approach to allow us to learn the lab material independently. I often found myself following the instructions in the textbook but needing to develop a complete understanding of the concepts the lab is related to. Thus, I felt confused and frustrated without clearly understanding the concepts. On the contrary, when I was in a math class learning about data management, the teacher would go through the lesson in a more direct and explicit approach, which allowed me to feel that I was beginning to understand the concepts of the topic presented. Thus, through my own learning experiences, I strongly support the main idea presented by the article.

The part where the article mentions how the human can be broken into two components: long-term memory and working memory. In particular, it also emphasizes the instructional importance of long-term memory, stating that the primary goal of instruction is to transfer knowledge and skills into long-term memory. However, the "worked-example effect" allows the student to balance long-term and short-term memories, which is an appropriate model for direct instruction.

During Lecture 1, we brought up the quote during a discussion prompt: "We have trained students to know that if they wait long enough, we'll give them the answer" \cite{lecture:1}. I find this quote relates to the "worked-example effect" because there are contrasting situations in which the connection may be in white, some in black, and some in gray. In particular, there are instances where students genuinely seek clarification or assistance, which can be a step in guiding students through their learning process. At the same time, there are instances where students may need more support from teachers for answers, which can degrade the development of independent thinking and problem-solving skills.

Reading through the article and relating my personal experiences and classroom discussions to the article allowed me to think about the following question: How can the ideas of fully-guided instruction and the "worked-example effect" evolve in the context of technology such as AI and VR? How can educators use these advancements to enhance student learning experiences? The proposed question allows educators to think about the impact of technological advances on instructional strategies and explore how these technologies complement or challenge the traditional approaches, as discussed in the article.

\cite{article:2}

\bibliographystyle{plain}
\bibliography{bibliography.bib}

\end{document}