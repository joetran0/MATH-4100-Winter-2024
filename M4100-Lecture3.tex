\documentclass[11pt]{article}
\usepackage[utf8]{inputenc}
\usepackage[margin=1in]{geometry}
\usepackage{amsfonts, amssymb, amsmath, amsthm, booktabs, hyperref, pgfplots, tikz, multicol, enumitem, xcolor}

\setlist{nosep, topsep=0pt}

\theoremstyle{theorem}\newtheorem*{task}{Task}
\theoremstyle{theorem}\newtheorem*{example}{Example}
\theoremstyle{theorem}\newtheorem*{answer}{Answer}
\theoremstyle{definition}\newtheorem*{solution}{Solution}
\theoremstyle{theorem}\newtheorem*{prompt}{Prompt}
\theoremstyle{theorem}\newtheorem*{question}{Question}

\begin{document}

\noindent \textbf{MATH 4100 Topics in Mathematics Education} \hfill \textbf{Lecture 3} \\
\textsc{Lecture} \hfill \textsc{Joe Tran}

\begin{prompt}
    I am a visual learner.
\end{prompt}

\section{Learning Styles}

The VARK model (Visual, Auditory, Reading, Kinesthetic) expands on the four areas of which student learns. The idea is that you learn best a certain way. Widely agreed to be a myth, but one that continues to sustain itself.
\begin{itemize}
    \item Husman (2018): Students took the VARK questionnaire and were given customized learning strategies. Most didn't actually study the way that questionnaires suggested. Those who applied [...]
    \item Knoll (2017): Students who thought they were visual learners believed thay would remember pictures better. Students who thought they  were auditory learners believed they would remember words better. No correlation with that they actually remembered. ``Learning style'' was just a preference, not an actual effect.
    \item Willingham (2018): ``It is much better to think of everyone having a toolbox of ways to think, and think to yourself, which tool is best?''
\end{itemize}

\section{Peer Feedback}

We use the RISE model
\begin{itemize}
    \item Reflect: reference specific parts of the writing (``I relate, concur, disagree with X because'')
    \item Inquire: ask a question (Have you considered looking at X and Y perspective)
    \item Suggest: offer specific ideas within the scope of the reading (You may wan to tweak X for Y effect).
    \item Elevate: propose specific ideas to expand the writing beyond its original purpose. (Perhaps you can discuss X in your reflection assignment if you also consider Y).
\end{itemize}

\section{Activity}

\begin{question}
    Which is larger? $x$ or $x^3$?
\end{question}

\begin{answer}
    It depends on the number system we are working with. For example, if we are talking about $\mathbb{N}$ and $\mathbb{Z}$, then there are instances where they are the same, i.e. $x = 0$ and $x = 1$, and we would agree that always, $x^3 \geq x$ for all $x \in \mathbb{Z}$. On the other hand, if we are working with $x \in \mathbb{F} = \mathbb{Q}, \mathbb{R}, \mathbb{C}$, then there are instances where $x^3 \geq x$ and instances where $x^3 \leq x$. In particular, $x \geq x^3$ on $x \in (-\infty, 1] \cup [0, 1]$, and $x^3 \geq x$ on $x \in [-1, 0] \cup [1, \infty]$
\end{answer}

Some considerations: What is $x$? What are some of the assumptions that we should consider.

\section{Multiple Representations}

There are five types of mathematical representations:
\begin{enumerate}
    \item Visuals (diagrams, images, graphs, drawings)
    \item Symbolic (numbers, variables, tables)
    \item Verbal (words and phrases---formal and non-formal)
    \item Contextual (math ideas in real life or in imagination)
    \item Physical (real objects, manipulatives)
\end{enumerate}

\subsection{Memory and Cognition}
\begin{itemize}
    \item Working memory vs long term memory
    \item Dual channel: visual/pictory vs auditory/Verbal
    \item Active processing: learning occurs when we organize material into a coherent structure
    \item Complex information can overwhelm the learner, information is not fully processed and doesn't reach long-term memory.
    \item Multiple representations convert information to visual form, transmit to visual channel, reduce need for high capacity of working memory.
    \item Representations themselves are passive---learner needs to active engage with them.
\end{itemize}

\subsection{Best Practices}
\begin{itemize}
    \item Encourage purposeful selection of representations.
    \item Engage in dialogue about explicit connections among representations.
    \item Alternate the direction of the connections made among representation.
\end{itemize}

\section{Case Study}

The third-grade class is responsible for setting up the chairs for the spring band concert. In preparation, they must determine the total number of chairs that will be needed and ask the school's engineer to retrieve that many chairs from the central storage area. Mr. Harris explains to his students that they need to set up 7 rows with 20 chairs in each row, leaving space for a centre aisle.

\section{Reflection Assignment}
\begin{itemize}
    \item 20\% of final grade
    \item Due on Monday, February 5th at 11:59 PM
    \item You can hand in one of the two reflection assignments up to one week late with no explanation or reason required.
\end{itemize}

Pick any mathematics topic/scenario/skill and reflect on the teaching and learning of this topic.

Examples: Adding fractions with common denominators, product rule, lining up your equal signs in a proof, rounding rules, private tutoring vs group instruction, defining variables, teaching to introverts, etc.

The purpose of these assignments is for you to demonstrate that you have fully engaged with the content of the class.

How long should my reflection be?
\begin{itemize}
    \item Average length is 4-5 single spaced and 8-10 double spaced pages.
    \item Quality over quantity.
\end{itemize}

To get a B:
\begin{itemize}
    \item Course materials materials to explain your chosen scenario.
    \item Focuses primarily on the education.
    \item Demonstrates a good amount of effort.
    \item Might contain some portions of opinion.
    \item Might be trying to convince me of your viewpoint.
\end{itemize}

To get an A:
\begin{itemize}
    \item Goes ``deep and not wide''.
    \item Looks at things from multiple perspectives and viewpoints.
    \item Uses some of the interlocuter ideas.
    \item Demonstrates substantive effort.
    \item Might explore multiple viewpoints before making a conclusion.
    \item Contains original and creative thoughts and ideas.
    \item All opinions are well defended and supported .
    \item Can be described using the verbs ``recommending'' or ``contrasting'' or ``hypothesizing'' or ``criticizing''
    \item Uses education topics strategically and effectively.
\end{itemize}

\end{document}