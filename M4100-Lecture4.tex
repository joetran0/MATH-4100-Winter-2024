\documentclass[11pt]{article}
\usepackage[utf8]{inputenc}
\usepackage[dvipsnames]{xcolor}
\usepackage[margin=1in]{geometry}
\usepackage{amsfonts, amssymb, amsmath, amsthm, booktabs, hyperref, pgfplots, tikz, xcolor, mathrsfs}

\theoremstyle{definition}\newtheorem{definition}{Definition}
\theoremstyle{definition}\newtheorem{question}{Question}
\theoremstyle{definition}\newtheorem*{solution}{Solution}
\theoremstyle{definition}\newtheorem{example}{Example}
\theoremstyle{definition}\newtheorem{notation}{Notation}
\theoremstyle{theorem}\newtheorem{theorem}{Theorem}
\theoremstyle{theorem}\newtheorem{corollary}{Corollary}
\theoremstyle{theorem}\newtheorem{lemma}{Lemma}
\theoremstyle{theorem}\newtheorem{proposition}{Proposition}
\theoremstyle{theorem}\newtheorem{prompt}{Prompt}

\newcommand{\A}{\mathcal{A}}
\newcommand{\B}{\mathcal{B}}
\newcommand{\C}{\mathbb{C}}
\newcommand{\CC}{\mathcal{C}}
\newcommand{\D}{\mathcal{D}}
\renewcommand{\d}{\delta}
\newcommand{\E}{\mathcal{E}}
\newcommand{\e}{\varepsilon}
\newcommand{\F}{\mathbb{F}}
\newcommand{\FF}{\mathcal{F}}
\newcommand{\G}{\mathcal{G}}
\renewcommand{\H}{\mathbb{H}}
\newcommand{\I}{\mathcal{I}}
\newcommand{\J}{\mathcal{J}}
\newcommand{\K}{\mathbb{K}}
\renewcommand{\L}{\mathscr{L}}
\newcommand{\M}{\mathcal{M}}
\newcommand{\N}{\mathbb{N}}
\renewcommand{\O}{\mathcal{O}}
\renewcommand{\P}{\mathcal{P}}
\newcommand{\Q}{\mathbb{Q}}
\newcommand{\R}{\mathbb{R}}
\renewcommand{\S}{\mathcal{S}}
\newcommand{\T}{\mathbb{T}}
\newcommand{\U}{\mathcal{U}}
\newcommand{\V}{\mathcal{V}}
\newcommand{\W}{\mathcal{W}}
\newcommand{\X}{\mathcal{X}}
\newcommand{\Y}{\mathcal{Y}}
\newcommand{\Z}{\mathbb{Z}}

\begin{document}

\noindent \textbf{MATH 4100 Topics in Mathematics Education} \hfill \textbf{Lecture 4} \\
\textsc{Lecture} \hfill \textsc{Joe Tran}

\begin{prompt}
    Multiplication makes things larger.
\end{prompt}

\color{teal}

The quote "Multiplication makes things larger" succinctly captures the essence of multiplication as a mathematical operation. When you multiply two numbers, you're essentially increasing the magnitude of one or both of them, resulting in a product that is larger than either of the original numbers. This principle applies not only to numbers but also to various contexts where multiplication is used, such as scaling in geometry or exponential growth in various natural and scientific phenomena. Overall, the quote encapsulates a fundamental truth about the nature and effect of multiplication.

\textbf{Positives:} The quote is clear and straightforward, making it easy to understand even for those who are not well-versed in mathematics. It can serve as a simple yet effective teaching tool for introducing the concept of multiplication to students, especially at a young age. The principle of multiplication applies across various disciplines and contexts, making the quote relevant beyond just mathematics.

\textbf{Negatives:} While the quote captures a fundamental truth about multiplication, it oversimplifies the concept and doesn't account for the nuances and complexities involved in mathematical operations. While the quote captures a fundamental truth about multiplication, it oversimplifies the concept and doesn't account for the nuances and complexities involved in mathematical operations. Without additional context, the quote may lead to misunderstandings or misinterpretations, especially if taken too literally or applied in inappropriate situations.

\textbf{Gray Area:} While multiplication generally results in larger values, its effect depends on the context in which it is applied. For instance, in certain mathematical operations or scenarios, multiplication might lead to a decrease in magnitude or have no effect at all. Multiplying by negative numbers or fractions can result in values that are smaller or negative, which contradicts the notion that multiplication always makes things larger. This highlights the need to consider the sign and magnitude of numbers involved. In geometric scaling, for instance, multiplying dimensions by a factor greater than 1 enlarges an object, but multiplying by a fraction shrinks it. This demonstrates the nuanced application of multiplication beyond simple numeric enlargement. While multiplication often leads to growth, in scenarios like exponential decay or diminishing returns, repeated multiplication can lead to a decrease in magnitude over time. Multiplication extends beyond real numbers into complex numbers, where multiplication can involve rotation and scaling in the complex plane, leading to outcomes beyond simple enlargement.

Overall, while the quote effectively communicates a basic concept of multiplication, it's important to recognize its limitations and to use it judiciously within the appropriate context.

\textbf{Question:} What does multiplication mean in terms of group theory or ring theory? Does it make stuff larger?

\color{black}

\begin{prompt}
    You cannot factor $4x + 1$
\end{prompt}

\color{teal}

It is possible technically speaking, if we consider complex numbers in this situation. $(2\sqrt{x})^2 - i^2 = (2\sqrt{x} - i)(2\sqrt{x} + i)$, or you may also just factor the 4 out, so that $4\left(x + \frac{1}{4}\right)$

\color{black}

\begin{prompt}[Rules That Expire, Karp et al, 2014]
    Overgeneralizing commonly accepted strategies, using imprecise vocabulary, and relying on tips and tricks that do not promote conceptual mathematical understanding can lead to misunderstandings later in students' math careers.
\end{prompt}

\begin{enumerate}
    \item Multiply everything inside the bracket by the term outside the bracket.
    \item Two negatives make a positive.
    \item When you multiply a number by ten, just add a zero to the end of the number.
\end{enumerate}

\color{teal}

Telling a student the quote you provided serves several purposes:

1. **Promoting Long-term Understanding:** The statement aims to encourage students to focus on understanding mathematical concepts rather than relying solely on shortcuts or memorization. By emphasizing the importance of conceptual understanding, educators hope to equip students with the skills necessary for deeper mathematical reasoning and problem-solving.

2. **Preventing Misconceptions:** By cautioning against overgeneralization and imprecise vocabulary, educators aim to prevent students from developing misconceptions that may hinder their progress in mathematics later on. This proactive approach helps students build a solid foundation of mathematical knowledge that can withstand more advanced concepts.

3. **Encouraging Critical Thinking:** The statement encourages students to critically evaluate the strategies and methods they encounter in mathematics. By promoting skepticism towards tips and tricks that lack conceptual grounding, educators foster a mindset of inquiry and reasoning, which are essential for mathematical proficiency.

Regarding the expiration of the method "When you multiply a number by ten, just add a zero to the end of the number," it's not that the method becomes invalid, but rather that it becomes too simplistic as students advance in their mathematical understanding. Here's why:

1. **Limited Applicability:** While the method works perfectly for basic multiplication by 10, it doesn't hold true for more complex multiplication or for understanding the principles behind multiplication. As students encounter larger numbers and more intricate mathematical operations, they need to develop a deeper understanding of place value and multiplication algorithms.

2. **Lack of Generalization:** This method only applies to specific cases (multiplying by 10) and doesn't provide a generalized understanding of multiplication. As students progress, they need to develop more versatile and robust strategies that can be applied across various mathematical contexts.

3. **Encouraging Dependency:** Relying solely on shortcuts like adding a zero can lead to a dependency on memorized rules without grasping the underlying principles. Students may struggle when faced with unfamiliar scenarios that require more flexible problem-solving skills.

Therefore, while the method may be useful as a introductory concept or mnemonic device, it's essential for students to move beyond such simplifications and develop a deeper, more flexible understanding of mathematical operations as they progress in their mathematical education.

\color{black}

\begin{enumerate}
    \item The absolute value is just the number.
    \item The most you can have is 100\% of something
    \item Use keywords to solve problems.
    \item FOIL
\end{enumerate}

\section{Announcements}

\begin{itemize}
    \item Weekly Readings Due and Feedback will be switched tonight.
\end{itemize}

\section{Language Matters}

What should you say instead of each of the following?
\begin{itemize}
    \item Solve an expression.
    \item Show all your steps.
    \item Get rid of the... [fraction, decimal, coefficient, like term, etc]
    \item Using the first letter of the word to describe the variable.
    \item 8 divides evenly into 24
    \item 2 plus 2 makes 4
    \item $\ln(e^x)$: The ln and the $e$ cancel to give $x$.
\end{itemize}

\section{Activity}

\begin{prompt}
    Calculate $24 \times 7$.
\end{prompt}

Their solution:
\begin{align*}
    24 \times 7 &= 20 \times 7 + 4 \times 7 = 140 + 28 = 168
\end{align*}

\begin{prompt}
    Given $f(x) = 10x + 110$, find $f(2)$.
\end{prompt}

Their solution:
\begin{align*}
    10 \times 2 = 20 + 110 = 130
\end{align*}

\begin{prompt}
    What goes in the $\square$?
    \begin{equation*}
        8 + 4 = \square + 5
    \end{equation*}
    What answers do you think that 6th graders provided?
\end{prompt}

Think about having left side of the equation ``the same'' as the right side of the equation. So if the left side makes 12, what value do we need to make the right side 12 as well?

\begin{prompt}
    What does this symbol $=$ mean?
\end{prompt}

The symbol means equality, representing that both sides of an equation are equivalent. It is also an equivalence relation when dealing with numbers or sets. The symbol would also indicate that this is the next step is equivalent to the previous step, and so on, until you reach the final answer. It is also meant to balance both the left side and the right side of an equation.
\begin{itemize}
    \item Relational: if it included the idea of equivalence or sameness.
    \item Operational: If it is included the idea of computing
    \item Other: If it includes ideas that were neither relational nor operational, typically defining the equal sign in terms of the name of the symbol.
\end{itemize}

Students in the study were asked
\begin{enumerate}
    \item Solve $10  =z + 6$
    \item Solve $c + c + 4 = 16$
    \item Solve $m + m + m + m = 12$.
    \item Cakes cost $c$ dollars each; brownies cost $b$ dollars each. Suppose I buy 4 cakes and 3 brownies. What does $4c + 3b$ mean?
\end{enumerate}

What are the similarities and differences in the meanings of the $=$ symbol in each of the following:
\begin{itemize}
    \item $\frac{24}{6} - 3 = 1$ (O)
    \item $\frac{d}{dx}(3x^2) = 6x$ (O)
    \item $19 = 10^2 - 9^2$ (R)
    \item $x^2 + x - 6 = (x - 2)(x + 3)$ (R)
    \item Solve $x^2 + x = 6$ (R)
    \item The volume of a cone is $V = \frac{1}{3}\pi r^2 h$ (R)
    \item The mean is $\frac{a + b}{2}$. (R)
\end{itemize}

\end{document}
