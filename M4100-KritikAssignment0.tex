\documentclass[11pt]{amsart}
\usepackage[utf8]{inputenc}
\usepackage[margin=1in]{geometry}
\usepackage{amsfonts, amssymb, amsmath, amsthm, booktabs, hyperref, pgfplots, tikz, xcolor}

\setlength{\parindent}{0pt}
\setlength{\parskip}{5pt}

\begin{document}

\textbf{MATH 4100 Topics in Mathematics Education} \hfill \textbf{Kritik Assignment 0} \\
\textsc{Kritik} \hfill \textsc{Joe Tran}

\quad The article demonstrates two different learning styles on how students can learn mathematics. The author surveyed two schools: Amber Hill School and Phoenix Park School. The Amber Hill School used a method that is commonly used traditionally, namely the textbook method. The Phoenix Park School used a technique that allowed the students to collaborate on projects. According to the article, students become more motivated whenever they are learning using the textbook, and you would hardly hear students complaining about the material they are learning. When students are open-ended, doing projects and collaborating with students, they do no work or do tiny chunks of work at a time.

\quad Throughout high school, I have mostly learned math through the traditional method. I liked how the conventional method was structured because the techniques taught during class can be used to solve exercises and answer questions. By understanding the math algorithms guided by the book, you can find ways to solve problems in general. The only downside of this method is that it is hard to find issues relating to mathematics that can be solved in the real world. Whenever students learn from the textbook, they tend to gain more knowledge and think more about the problems and exercises related to a specific concept. When the students work collaboratively, they believe more and apply the concepts to applications. 

\quad After reading the article, I would like to think about what would happen in a high school math class, where a teacher would implement both the use of learning from the textbook and a system where open-ended projects are also supported in a class. Through this approach, the students can have a well-balanced learning experience that promotes knowledge through procedural methods, thinking regarding problem-solving questions, project applications, and communication and collaboration with students. Furthermore, the students would have to deal with the negatives if the teacher implements a learning environment where open-ended problems and traditional learning experiences are supported.

\end{document}

-12 words