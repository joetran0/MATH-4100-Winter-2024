\documentclass[11pt]{article}
\usepackage[utf8]{inputenc}
\usepackage[margin=1in]{geometry}
\usepackage{amsfonts, amssymb, amsmath, amsthm, booktabs, hyperref, pgfplots, tikz, multicol, enumitem, xcolor}

\setlist{nosep, topsep=0pt}

\theoremstyle{theorem}\newtheorem*{task}{Task}
\theoremstyle{theorem}\newtheorem*{example}{Example}
\theoremstyle{definition}\newtheorem*{solution}{Solution}
\theoremstyle{theorem}\newtheorem*{prompt}{Prompt}

\begin{document}

\noindent \textbf{MATH 4100 Topics in Mathematics Education} \hfill \textbf{Lecture 2} \\
\textsc{Lecture} \hfill \textsc{Joe Tran}

\begin{prompt}
    Memorizing mathematics is bad.
\end{prompt}

\textcolor{blue}{I disagree with the prompt because there are certain areas of mathematics in which memorizing is needed. In particular, in analysis, you are basically required to know what definitions, axioms, and theorems to be able to apply it to a problem. It is the same with other areas that involve computation. In a high school level, say one student was to memorize the quadratic formula to be able to solve problems relating to quadratic equations.}

A facilitator asks questions to provoke some else to elaborate on an idea they are already expressing. An interlocutor analyses responses, points out where disagreements lie, and looks for reasons for/against perspectives.

People mostly agree, so you don't dig deeper. Facilitators are great at extending discussion when this occurs, but it does not break the stagnancy. Interlocutors can facilitate [...]

\begin{task}
    \begin{itemize}
        \item[(a)] If 1 ticket costs \$0.50, how much do 2 tickets cost?
        \item[(b)] If I have \$20 and I buy 25 tickets and then 12 tickets, how much change will I receive.
        \item[(c)] Which ticket is the best/worst deal?
        \item[(d)] How would you suggest they change their prices?
    \end{itemize}
    \textcolor{red}{\begin{center}
        \fbox{\fbox{\parbox{0.25\linewidth}{\centering 1 ticket = \$0.50 \newline 12 tickets = \$5.00 \newline 25 tickets = \$10.00 \newline 50 tickets = \$25.00 \newline 120 tickets = \$50.00}}}
    \end{center}}
\end{task}

\begin{itemize}
    \item[(a)] 1 ticket costs \$0.50, so 2 tickets would cost \$1.00.
    \item[(b)] Buying the 25 ticket option would cost \$10.00 and buying the 12 ticket option would cost \$5.00, so the total cost for 37 tickets would be \$15.00, so the change left over would be \$5.00.
    \item[(c)] Best deal would be the cost for 25 tickets, and the worse deal would be the cost for 120 tickets.
    \item[(d)] Adjust the prices so that the more tickets that are bought, the cost of the tickets should decrease.
\end{itemize}

\noindent \textbf{Task Analysis Guide}

\begin{example}[Memorization]
    What are the decimal and percent equivalents for fractions $\frac{1}{2}$ and $\frac{1}{4}$.
\end{example}

\begin{example}[Procedures without Connections]
    Convert the fraction $\frac{3}{8}$ to a decimal and a percent.
\end{example}

\begin{example}[Procedures with Connections]
    Using a $10 \times 10$ grid, identify the decimal and percent equivalents of $\frac{3}{5}$.
\end{example}

\begin{example}[Doing Mathematics]
    Shade 6 small squares in a $4 \times 10$ rectangle. Using the rectangle, explain how to determine each of the following.
    \begin{itemize}
        \item[(a)] The percent of area that is shaded.
        \item[(b)] The decimal part of area is shaded.
        \item[(c)] The fractional part of area that is shaded.
    \end{itemize}
\end{example}

\noindent \textbf{Decline of a Task}

Routinizing problematic aspects of the task. Shifting the emphasis from meaning, concepts, or understanding to the correctness or completeness of the answer. Providing insufficient time to wrestle with the demanding aspects of the task or so much time that students drift into off-task behavior. Engaging in high level cognitive activities is prevented due to classroom management problems. Selecting a task that is inappropriate for a given group of students. Failing to hold students accountable for high-level tasks.

\noindent \textbf{Maintenance of a Task}

Scaffolding of student thinking and reasoning. Providing a means by which students can monitor their own progress. Modeling of high-level performance by teacher or capable students. Pressing for justifications, explanations and/or meaning through questioning, comments, and/or feedback. Selecting tasks that build on students' prior knowledge. Drawing frequent conceptual connections. Providing sufficient time to explore.


\end{document}