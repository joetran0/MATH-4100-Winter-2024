\documentclass[11pt]{amsart}
\usepackage[utf8]{inputenc}
\usepackage[margin=1in]{geometry}
\usepackage{amsfonts, amssymb, amsmath, amsthm, booktabs, hyperref, pgfplots, tikz, xcolor, multicol}

\setlength{\parindent}{0pt}
\setlength{\parskip}{5pt}

\begin{document}

\textbf{MATH 4100 Topics in Mathematics Education} \hfill \textbf{Lecture 1} \\
\textsc{Lecture} \hfill \textsc{Joe Tran}

\section{Course Outline}

\subsection{Content} 

We will be covering the following in this class.
\begin{itemize}
    \item Wait time
    \item Classifying mathematical tasks
    \item Asking mathematical questions
    \item Multiple representatives
    \item Manipulatives
    \item Symbols, notations, and communication
    \item Educational technology
    \item Math anxiety
    \item Authentic examples
    \item Assessments, worked examples and diagnostics
\end{itemize}

\subsection{Grading}

The following is the grading scheme for the class.

\begin{itemize}
    \item 20\% Weekly Readings (5 readings, 4\% each)
    \item 45\% Reflection Assignments
    \begin{itemize}
        \item 20\% Reflection Assignment 1
        \item 25\% Reflection Assignment 2
    \end{itemize}
    \item 10\% Participation
    \item 25\% Communication Project
\end{itemize}

Reading \#0, a calibration activity will show you how the system works, will be due next Monday at 11:30 AM.

You are expected to use the course materials to analyze and/or explain something you have experienced or witnessed as a student and/or instructor.

To obtain an A in this class, need to go above and beyond for the case-by-case analyses.

Submit all assignments on eClass, which will all be online.

\subsection{Expectations}

Expected to...
\begin{itemize}
    \item Attend every class and take notes.
    \item Participate in each activity.
    \item Use the course materials heavily in the assessments.
    \item Ask frequent questions.
    \item Disagreements with each other
    \item Listen and respond to feedback.
    \item Take advantage of the opportunity this course provides in enhancing your unique skill set.
    \item 3 hours of class, 6 hours of work outside.
    \item Solicit and respond to feedbackModel good pedagogical practice
    \item Encourage you to personalize your learning
    \item Provide constructive feedback
    \item Have high expectations
\end{itemize}

\section{Typical Class Structure}

\subsection{Discussion Prompt}

A prompt on the board/screen when you arrive in class and there will be group discussions/breakout rooms.

\subsection{Math Activity}

A math problem to work on in groups and you'll discuss how the readings apply to the problem.

\subsection{Education Topic(s)}

One or two math education topic with activities and discussions

\subsection{Case Study}

A specific educational problem to discuss and debate.

\section{Discussion Prompt}

\subsection{Prompt} 

What was your best learning experience? (i.e., best teacher, best class, etc.)

\subsection{Prompt}

What was your worst learning experience?

\subsection{Prompt}

``We have trained students to know that if they wait long enough, we'll give them the answer.'' What are your thoughts on this quote?

\begin{itemize}
    \item In high school, students would wait for a long time for their teachers to provide them with an answer because the student would not know how to approach it at all.
    \item In university level math course, students would get stuck on a problem, so they would circle it and eventually come back to the question.
\end{itemize}

\subsection{Could put on a reflection}

\begin{itemize}
    \item How the style of a textbook can affect students learning. (Some textbooks can have no solutions, partial answers, answers to even problems, or full solutions)
    \begin{itemize}
        \item When having full solutions, there are some positive aspects and negative aspects.
        \item When having partial solutions, there are more positive aspects than negative aspects.
        \item When having no solutions, there are more positive aspects than negative aspects, but can also have more negative aspects than positive aspects as well.
    \end{itemize}
\end{itemize}

\section{Wait Time}

If teachers can increase their pauses after:
\begin{itemize}
    \item They ask a question, and
    \item After a student responds,
\end{itemize}
there are significant positive changes. The threshold waiting time is 2.7 seconds, above which there are significant improvements and below which there is little effect. (1986, Rowe)

What do you think are the specific benefits to wait time? \emph{The benefits of wait time include digesting the information that is being provided to the student, allowing the student to catch up in the moment you are currently pausing, or to think about the question and be able to provide an answer.}
\begin{itemize}
    \item Increases the quality of the answer from wait time.
    \item Take a moment to refine what you are about to say.
    \item Allows more students to be more heard.
    \item More time for students to think with each other, rather than having the teacher replying.
    \item Have students start reflecting on the answer that they know is correct, and be able to explain to others why their answer is correct.
    \item A right amount of wait time would allow more participation allotted.
\end{itemize}

Research supported benefits to students:
\begin{enumerate}
    \item Length of student responses increases 300\%-700\%
    \item More student responses are supported by evidence.
    \item Number of questions asked by students increases.
    \item Student-student exchanges increases.
    \item Fewer `I don't know' responses.
    \item Less disciplinary issues.
    \item Increased variety of students participating.
    \item Student confidence increases.
\end{enumerate}

Research supported benefits to teachers:
\begin{enumerate}
    \item Teachers demonstrate greater flexibility.
    \item Teachers ask less questions, but the questions are higher-order (inviting elaboration or opposing viewpoints).
    \item ``Invisible students'' become visible.
\end{enumerate}

\section{The Math Part}

Given an isosceles triangle, $\triangle ABC$ with $\angle ABC = 74^\circ$, find the reflex angle of $\angle ACB$.
\begin{itemize}
    \item[(a)] $74^\circ$
    \item[(b)] $254^\circ$
    \item[(c)] $286^\circ$
    \item[(d)] $307^\circ$
\end{itemize}

\begin{proof}
    We claim that the reflex angle of $\angle ACB$ is $286^\circ$. Because $\triangle ABC$ is an isosceles triangle, then $\angle ABC = \angle ACB = 74^\circ$. Then because $\angle ABC + x = 360^\circ$ is the complete angle of a circle, then $x = 286^\circ$.
\end{proof}

Students may answer A because because they might've only found that $\angle ABC = \angle ACB$, but might not have found the \emph{reflex} angle as asked.

Students may answer B by adding the angle $\angle ACB = 74^\circ$ and $180^\circ$.

Students may answer D because they might have not considered what sides are the same, so they may have assumed that the other angles are the same, but halved, i.e. $74^\circ + 53^\circ + 53^\circ$, and then $360^\circ - 53^\circ = 307^\circ$. Misunderstood how the triangle is given.

\end{document}