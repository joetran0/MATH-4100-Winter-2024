\documentclass{amsart}
\usepackage[utf8]{inputenc}
\usepackage{amsfonts, amssymb, amssymb, amsthm, booktabs, hyperref, pgfplots, tikz, xcolor}

\title{MATH 4100 Topics in Mathematics Education (Lecture 5)}
\author{Joe Tran}
\date{\today}

\theoremstyle{theorem}\newtheorem{prompt}{Prompt}

\begin{document}

\maketitle

\begin{abstract}
    Discussion prompt on eliminating homework. Homework best practices. Mathematical task on convincing someone that $3/7$ is larger than $4/11$. Benefits of diagnostic questions. Criticism of multiple choice questions.
\end{abstract}

\section{Discussion Prompt}

\begin{prompt}
    We should eliminate homework.
\end{prompt}

\begin{itemize}
    \item \textbf{No, we should not eliminate homework.}
    \begin{itemize}
        \item What if the test was homework based, and on the test, you are stuck on the question, and the teacher tells you, well if you did the questions then you would know how to do it.
        \item Parents: More homework will allow students to improve their grades.
        \item Teachers: Needs the extra practice for the students to do well in this class.
        \item Homework is for your learning, knowledge checking, and conceptualizing with the ideas.
    \end{itemize}
    \item \textbf{Yes, we should eliminate homework.}
    \begin{itemize}
        \item Students may make an excuse that they won't have enough time to do homework, but use that as an excuse to play video games or waste time scrolling through TikTok, or just napping.
        \item Students: I do not have the motivation to do my homework, I am just going to skip, no big deal, but why...?
    \end{itemize}
    \item \textbf{Depends}
    \begin{itemize}
        \item It depends on what kind of “homework” are we talking about here? Homework assignments? Or homework for your own benefits?
        \item What if you get Chegg or AI to do it?
    \end{itemize}
\end{itemize}

\section{Homework Best Practices}
\begin{itemize}
    \item Assign homework with clear purpose (Shellard 2005)
    \item Communicate expectations to students/parents (Horowitz 2005)
    \item Use to enrich classroom curriculum (Plato 2000)
    \item Match it to students' skills (Marzano 2007)
    \item Keep drill work to a minimum (Bluestein 2006)
    \item Be intentional (McBeath 1996)
    \item Explicitly teach needed skills (Paulu 1998)
    \item Monitor the amount assigned (Hancock 2001)
    \item Coordinate with other teachers (Simplicio 2005)
    \item Don't use as a punishment (Silvis 2001)
    \item Don't assign at the end of the class (Brewster 2000)
    \item Flexible homework completion policy (Battle-Bailey 2003)
    \item Provide Feedback (Walberg 2004)
\end{itemize}

\section{Mathematical Task}

\begin{prompt}
    How would you convince someone that $\frac{3}{7}$ is bigger than $\frac{4}{11}$.
\end{prompt}

\begin{itemize}
    \item \textbf{Students who would answer this correctly}
    \begin{itemize}
        \item Find the common denominator between $\frac{3}{7}$ and $\frac{4}{11}$.
        \item Draw a pie chart.
        \item Convert it to something that students would be able to see easily.
        \item Use a calculator.
    \end{itemize}
    \item \textbf{Students who would answer this incorrectly}
    \begin{itemize}
        \item They might just look at the 3 and the 4 and argue that 4 would be larger than 3 so $\frac{4}{11}$ is larger than $\frac{3}{7}$, 
        \item They could also argue that because 11 is larger than 7, so $\frac{4}{11}$ is larger than $\frac{3}{7}$.
    \end{itemize}
\end{itemize}

Why would this be a good question:
\begin{itemize}
    \item Encourages students to think
    \item Encourages students to take time to articulate.
    \item Provoke discussions and disagreements.
\end{itemize}
This is an ideal in a \emph{summative assessment}.

\section{Benefits of Diagnostics}

For any assessment strategy---formative or summative---to work, students must actively and honestly participate. \emph{What are some factors that prevent this?}
\begin{itemize}
    \item Fear of mistakes
    \item Students opting out
    \begin{itemize}
        \item[$\bigstar$] A small discussion group surrounded by many sleepy onlookers.
    \end{itemize}
    \item Finding comfort in one correct answer.
\end{itemize}

What makes a good diagnostic assessment? (Barton, 2018)
\begin{itemize}
    \item It should be clear and unambiguous
    \item It should test a single skill/concept.
    \item Students should be able to answer it in less than 10 seconds
    \item You should learn something from each incorrect response without the student needing to explain.
    \item It cannot be answered correctly while still holding a key misconception.
\end{itemize}

\begin{prompt}
    Consider the power series $\sum_{n = 1}^{\infty} \frac{(x - 4)^n}{n3^n}$.
    \begin{itemize}
        \item[(a)] Use the ratio test to identify which of the conditions on the values of $x$ listed below will ensure the power series converges. What is the radius of convergence of this power series.
    \end{itemize}
    \textsc{Options:} (A) $|x - 4| < \infty$, (B) $|x - 4| < 3n$, (C) $|x - 4| < 1$, (D) $|x - 4| > 1$, (E) $|x - 4| > 0$.
\end{prompt}

\section{Criticism of MC Questions}
\begin{itemize}
    \item Students can guess
    \item Students can work backwards to get the answer.
    \item We do not have the chance to see what other answers students would have came up with.
    \item These are trick questions.
    \item The exam is not multiple choice.
    \item They are easier than ``normal'' questions.
    \item The distractions are dangerous---they can cause students to develop misconceptions.
\end{itemize}

The most common misconceptions we believe our students to hold may not be the same as the ones they do in fact hold---the curse of knowledge.
        

\end{document}